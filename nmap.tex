\documentclass{article}

\title{Notes on Nmap functionality}
\author{Torres}
\date{\today}

\begin{document}
\maketitle
\section{The Network Mapper}

Network mapper. It's a C tool that scan networks sending
raw IP packages to hosts and return open ports, information
on the services running on those open ports, operating
system running on the host, etc.

Most hackers use this tool because it gives you absurd
advantage in the recognition phase (it's much easier to
exploit a system if you know the vulnerabilities it probably
has, and you'll only know that if you know at least the
services running on the ports of the system).

\section{TCP Syn Scan}

TCP Syn scan will determine the status of ports in the remote
target. RFC 793 defines the required behavior of any TCP/IP device
is that an incoming connection request begins with a SYN package,
which in turn must be followed by a SYN/ACK\footnote{Syn stands
for "synchronize", and "ACK" for "acknowledge"} packet from
the receiving service.

The main characteristic of Syn scan is that it doesn't complete
the three way handshake. The SYN/ACk package will never reach the
host system, and therefore you don't need to waste more time
connecting to a port.

The scanning rate is extremely fast because no time is wasted
completing the handshake or tearing down the connection. This
technique allows an attacker to scan through stateful firewalls
due to the common configuration that TCP SYN segments for a new
connection will be allowed for almost any port. TCP SYN scanning
can also immediately detect 3 of the 4 important types of port
status: open, closed, and filtered.

The port is open if the Syn/ACK package is sent back to Nmap's
Syn package. Syn scans are not completely anonymous, with a
decent network administration the packages nmap sent to the host
can be found and their source (your computer) is detectable.

\section{TCP Connect scan}

The main difference between Connect and Syn scans is that the
Nmap completes the three way handshake, it completes the connection
between your machine and the host.

\end{document}
